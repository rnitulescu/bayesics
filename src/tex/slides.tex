%%%%%%%%%%%%%%%%%%%%%%%%%%%%%%%%%%%%%%%%
%% Slides : Bayes basics ("bayesics") %%
%%%%%%%%%%%%%%%%%%%%%%%%%%%%%%%%%%%%%%%%

%% PREAMBLE
%% Define document class and basic options
\documentclass{beamer}
%\setlength{\parindent}{0pt}

%% Load packages
\usepackage{palatino}
\usepackage{amsfonts}
\usepackage{amsmath}
%\usepackage{url}
\usepackage{hyperref}
%\usepackage{listings}
\usepackage{verbatim}
\usepackage[utf8]{inputenc} %% For french

\hypersetup{
	colorlinks=true,
	linkcolor=blue,
	citecolor=red,
	filecolor=blue,
	urlcolor=blue
}

\usetheme{Madrid}

%% Basic info
\title{Introduction à la pensé et aux méthodes bayésiennes}
%\subtitle{}
\author{Roy Nitulescu\inst{1}}

\institute
{
    \inst{1}%
    CITADEL\\
    CR-CHUM
}

\date[UdeM, Sept. 22, 2021]{Université de Montréal, Sept. 22, 2021}

\AtBeginSection[]
{
    \begin{frame}
        \frametitle{Table des matières}
        \tableofcontents[currentsection]
    \end{frame}
}

\AtBeginSubsection[]
{
    \begin{frame}
        \frametitle{Table des matières}
        \tableofcontents[currentsubsection]
    \end{frame}
}


%% BEGIN DOCUMENT
\begin{document}

%%%%
%% Slides
%%%%

\frame{\titlepage}

\begin{frame}
    \frametitle{Épigraphe}
    ``La probabilité est le concept le plus important de la science moderne,
    d'autant plus que personne n'a la moindre idée de ce qu'elle signifie.'' -- Bertrand Russell, 1929
\end{frame}


\begin{frame}
    \frametitle{Préambule}
%    \begin{itemize}
%      \item This 3-hour lecture will be broken down into 3 modules
%      \item Each module will last 50 minutes
%      \begin{itemize}
%        \item 30 minutes of lecture time
%        \item 10 minutes for exercises
%        \item 10 minutes for discussing the exercises
%      \end{itemize}
%      \item There will be two 15-minute breaks between the modules
%      \item I expect that all students have already installed R on their computer and tested that it works
%    \end{itemize}
    
%    \vfill
    
    Le code source pour cette présentation ce trouve içi:\\
    \url{https://github.com/rnitulescu/bayesics}
\end{frame}


\begin{frame}
    \frametitle{Table des matières}
    \tableofcontents
\end{frame}


%%%%
%% MODULE 1
%%%%

\section{Motivation}

\begin{frame}
    \frametitle{La dualité du concept de probabilité}

    De par son origine, le concept à une \textbf{dualité}\footnote{
    Hacking, Ian (1975). \emph{The emergence of probability:
    A philosophical study of early ideas about probability,
    induction and statistical inference}. Cambridge University Press.
    }

    \vfill

    \begin{enumerate}
      \item \textbf{Aléatoire}
        \begin{itemize}
          \item Fréquence (la limite d'une fréquence relative dans une séquence infinie)
          \item Propensité (une propriété intrinsèque d'objets ou de situations)
        \end{itemize}
      \item \textbf{Épistémique}
        \begin{itemize}
          \item Logique (le degré de soutien ou de confirmation qu'un élément d'évidence confère à une hypothèse donnée)
          \item Subjectif (le degré de croyance subjectif)
        \end{itemize}
    \end{enumerate}    
\end{frame}

\begin{frame}
    \frametitle{Plusieurs interprétations}
    Donc, plusieurs \textbf{interprétations} de la probabilité ont été proposées. Mais, comment les évaluons-nous?
    
    \vfill

    Des \textbf{critères} d’évaluation proposés: ``admissibility'', ``ascertainability'', ``applicability''\footnote{
    Salmon, W. C. (1967). \emph{The foundations of scientific inference}.
    University of Pittsburgh Press.
    }

    \vfill

    Mais \textbf{aucune} des interprétations ne répond à tous ces critères.

    \vfill

    Pour plus de détails, voir aussi:
    \href{https://plato.stanford.edu/entries/probability-interpret/}{https://plato.stanford.edu}
\end{frame}

\begin{frame}
    \frametitle{Plusieurs systèmes de probabilité}
    Plusieurs \textbf{systèmes} axiomatiques de probabilité ont été développés.

    \begin{itemize}
      \item Kolmogorov (fréquence)
      \item Carnap (logique)
      \item De Finetti (subjectif/personnaliste)
    \end{itemize}

    \vfill

    En pratique, ces systèmes sont souvent plus semblables que différents.

    \vfill

    Il y a même le \textbf{Théorème de Bernstein-von Mises} qui relient le système fréquentiste
    d’inférence au système bayésien (en bref, sous certaines hypothèses, asymptotiquement,
    les intervalles de confiances sont équivalents aux intervalles crédibles bayésiennes).
    \footnote{
      \url{https://en.wikipedia.org/wiki/Bernstein-von_Mises_theorem}
    }
\end{frame}








\section{Intuition}

%\subsection{}

%% mostly examples just using discrete event probabilities

\section{Théorie}

%% formally present bayes theoerem even tough we sort of saw it informally in above section
%% the big step here is De Finetti's theorem, since it allows us to do some cool modeling, etc.
%% Present the model of the mean with know or unknown variance, etc. (do we need to go that far??)


\section{Applications}

%% the bulk of content
%% have the sex ratio example from Aki Vehtari, etc.


\section{Conclusion}

\begin{frame}
    \frametitle{Références et lectures suggérées}

    Philosophie:
    \begin{itemize}
      \item Hacking, I. (1975). \emph{The emergence of probability:
            A philosophical study of early ideas about probability,
            induction and statistical inference}. Cambridge University Press.
      \item Salmon, W. C. (1967). \emph{The foundations of scientific inference}.
            University of Pittsburgh Press.
    \end{itemize}

    Méthodes bayésiennes:
    \begin{itemize}
      \item Kruschke, J. K. (2013). Bayesian estimation supersedes the t test.
            \emph{Journal of Experimental Psychology: General}, \emph{142}(2), 573.
      \item Kruschke, J. K., \& Vanpaemel, W. (2015). Bayesian estimation in hierarchical models.
            \emph{The Oxford handbook of computational and mathematical psychology, 279}.
      \item Kruschke, J. K.(2015). \emph{Doing Bayesian data analysis: A tutorial with R, JAGS, and Stan}.
            Academic Press.
      \item Gelman, A., Carlin, J. B., Stern, H. S., \& Rubin, D. B. (1995).
            \emph{Bayesian data analysis}. Chapman and Hall/CRC.
    \end{itemize}
\end{frame}


%% END DOCUMENT
\end{document}

