%%%%%%%%%%%%%%%%%%%%%%%%%%%%%%%%%%%%%%%%
%% Slides : Bayes basics ("bayesics") %%
%%%%%%%%%%%%%%%%%%%%%%%%%%%%%%%%%%%%%%%%

%% PREAMBLE
%% Define document class and basic options
\documentclass{beamer}
%\setlength{\parindent}{0pt}

%% Load packages
\usepackage{palatino}
\usepackage{amsfonts}
\usepackage{amsmath}
%\usepackage{url}
\usepackage{hyperref}
%\usepackage{listings}
\usepackage{verbatim}
\usepackage[utf8]{inputenc} %% For french

\hypersetup{
	colorlinks=true,
	linkcolor=blue,
	citecolor=red,
	filecolor=blue,
	urlcolor=blue
}

\usetheme{Madrid}

%% Basic info
\title{Introduction à la pensé et aux méthodes bayésiennes}
%\subtitle{}
\author{Roy Nitulescu\inst{1}}

\institute
{
    \inst{1}%
    CITADEL\\
    CR-CHUM
}

\date[UdeM, Sept. 22, 2021]{Université de Montréal, Sept. 22, 2021}

\AtBeginSection[]
{
    \begin{frame}
        \frametitle{Table des matières}
        \tableofcontents[currentsection]
    \end{frame}
}

\AtBeginSubsection[]
{
    \begin{frame}
        \frametitle{Table des matières}
        \tableofcontents[currentsubsection]
    \end{frame}
}


%% BEGIN DOCUMENT
\begin{document}

%%%%
%% Slides
%%%%

\frame{\titlepage}


\begin{frame}
    \frametitle{Préambule}
%    \begin{itemize}
%      \item This 3-hour lecture will be broken down into 3 modules
%      \item Each module will last 50 minutes
%      \begin{itemize}
%        \item 30 minutes of lecture time
%        \item 10 minutes for exercises
%        \item 10 minutes for discussing the exercises
%      \end{itemize}
%      \item There will be two 15-minute breaks between the modules
%      \item I expect that all students have already installed R on their computer and tested that it works
%    \end{itemize}
    
%    \vfill
    
    Le code source pour cette présentation ce trouve içi:\\
    \url{https://github.com/rnitulescu/bayesics}
\end{frame}


\begin{frame}
    \frametitle{Table des matières}
    \tableofcontents
\end{frame}


%%%%
%% MODULE 1
%%%%

\section{Motivation}

\subsection{La dualité du concept de probabilité}

\begin{frame}
    \frametitle{La dualité du concept de probabilité}
    ...
\end{frame}


\section{Intuition}

%% mostly examples just using discrete event probabilities

\section{Théorie}

%% formally present bayes theoerem even tough we sort of saw it informally in above section
%% the big step here is De Finetti's theorem, since it allows us to do some cool modeling, etc.
%% Present the model of the mean with know or unknown variance, etc. (do we need to go that far??)


\section{Applications}

%% the bulk of content
%% have the sex ratio example from Aki Vehtari, etc.


%% END DOCUMENT
\end{document}

